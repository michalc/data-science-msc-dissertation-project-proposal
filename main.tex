\newcommand*{\examnumber}{B135653}
\newcommand*{\field}{Reproducible research and \\resilient supply chains in OpenTTD}
\newcommand*{\tutor}{Bob Fisher}
\newcommand*{\supervisor}{Michael Herrmann}
\newcommand*{\KEYWORDS}{\field, \supervisor, \tutor, School of Informatics, University of Edinburgh}

%
%                       This is a basic LaTeX Template
%                       for the Informatics Research Review

\documentclass[a4paper,11pt]{article}
% Add local fullpage and head macros
\usepackage{head,fullpage}     
% Add graphicx package with pdf flag (must use pdflatex)
\usepackage[pdftex]{graphicx}  
% Better support for URLs
\usepackage{url}
% Date formating
\usepackage{datetime}
% For Gantt chart
\usepackage{pgfgantt}
\usepackage{xcolor}
\usepackage{gnuplottex}
\usepackage[utf8]{inputenc}

\definecolor{hyperlinkColor}{HTML}{0D3B68}
\usepackage[shrink=20,stretch=20]{microtype}
\usepackage{siunitx}
\sisetup{detect-all
  ,group-minimum-digits=3% Western convention is groups of 3 digits
  ,mode = text
%   ,text-font-command = \liningroman
}
\usepackage{xurl}
\usepackage{hyperref}
\hypersetup{%
   pdfauthor=\texorpdfstring{\examnumber}{\examnumber}%
  ,pdftitle=\texorpdfstring{\field}{\field}%
  ,pdfsubject=\texorpdfstring{\field}{\field}%
  ,pdfkeywords=\texorpdfstring{\KEYWORDS}{\KEYWORDS}%
  ,linktoc=all%
  ,colorlinks=true%\ifdefstring{\expandafter\docUsage}{print}{false}{true}
  ,linkcolor=hyperlinkColor
  ,linkbordercolor=hyperlinkColor% internal hyperlink border colour
  ,urlcolor=hyperlinkColor
  ,urlbordercolor=hyperlinkColor% external hyperlink border colour
  ,citecolor=hyperlinkColor
  ,citebordercolor=hyperlinkColor% internal citation border colour
  ,pdfborderstyle={/S/U/W 1.5}% border style will be an underline of width 1.5pt
}

\usepackage{cleveref}
\usepackage[shortcuts]{extdash}
% gives \=/ for non-breaking hyphen
% and \-/ to allow the word before the hyphen to be hyphenated
\newdateformat{monthyeardate}{%
  \monthname[\THEMONTH] \THEYEAR}

\parindent=0pt          %  Switch off indent of paragraphs 
\parskip=5pt            %  Put 5pt between each paragraph  
\Urlmuskip=0mu plus 1mu %  Better line breaks for URLs


%                       This section generates a title page
%                       Edit only the following three lines
%                       providing your exam number, 
%                       the general field of study you are considering
%                       for your review, and name of IRR tutor
\usepackage{floatpag}
\floatpagestyle{empty}

\begin{document}
\begin{minipage}[b]{110mm}
        {\Huge\bf School of Informatics
        \vspace*{17mm}}
\end{minipage}
\hfill
\begin{minipage}[t]{40mm}               
        \makebox[40mm]{
        \includegraphics[width=40mm]{crest.png}}
\end{minipage}
\par\noindent
    % Centre Title, and name
\vspace*{2cm}
\begin{center}
        \Large\bf Informatics Project Proposal \\
        \Large\bf \field
\end{center}
\vspace*{1.5cm}
\begin{center}
        \bf \examnumber\\
        \monthyeardate\today
\end{center}
\vspace*{5mm}

%
%                       Insert your abstract HERE
%                       
\begin{abstract}
OpenTTD \cite{openttd} is an open source real time strategy (RTS) supply chain simulation computer game that allows the creation of so-called AIs - computer players that build supply chains to compete with human players. This project is two-fold - extend OpenTTD so it can be used for reproducible experiments into such AIs, and to use these extensions to investigate the relationships between robustness and other properties of the supply chains. This adds to the body of research into supply chains affected by events such as coronavirus (COVID-19) pandemic or the 2021 Suez Canal obstruction by the container ship Ever Given.
\end{abstract}

\vspace*{1cm}

\vspace*{3cm}
Date: \today

\vfill
{\bf Tutor:} \tutor\\
{\bf Supervisor:} \supervisor
\newpage

%                                               Through page and setup 
%                                               fancy headings
\setcounter{page}{1}                            % Set page number to 1
\footruleheight{1pt}
\headruleheight{1pt}
\lfoot{\small School of Informatics}
\lhead{Informatics Research Review}
\rhead{- \thepage}
\cfoot{}
\rfoot{Date: \date{\today}}
%


\section{Motivation}

\begin{figure}[h]
\centering
\includegraphics[width=\textwidth]{transport-tycoon-screenshot.png}
\caption{Screenshot of an OpenTTD game showing part of a transport network with multiple industries and mechanisms of transport. One of the trains has broken down, which not only prevents its cargo from reaching its destination, but it also prevents other trains from progressing along that track; OpenTTD has built-in ways of penalising non-robustness.}
\label{fig:network}
\end{figure}

OpenTTD is an open source business simulation game based on the 1990s game Transport Tycoon Deluxe. The aim of the game is to create a business that makes money from the transport of passengers and goods from one place to another, by constructing and using a network of road vehicles, trains, planes, or ships - also known as supply chains. An example of part of such a network can be seen in Figure \ref{fig:network}.

OpenTTD allows the implementation of so-called AIs, custom computer-based players that create their own supply chains to compete with human players. Over 50 such AIs have been created \cite{openttdAIs}. However, since OpenTTD is primarily a computer game rather than a tool for research, it is lacking in components for running reproducible experiments with such AIs and extracting data from them \cite{openttdNoHeadless}. The first aim of this project is to extend OpenTTD to allow such reproducible experiments.

The second aim of this project is to use OpenTTD and any extensions to construct an AI that allows the investigation of the links between robustness of such supply chains and other properties of the networks. While the focus of this project is OpenTTD, results could be compared with investigations of real-world supply chains, and so adds to the body of research of such supply chains, and so could be used to inform government and corporate policies.

\subsection{Problem Statement}

The first problem is the problem than OpenTTD is not suitable to run reproducible experiments. The work in this project is to change OpenTTD to make it possible to do so. For example, after initial investigation, it appears that OpenTTD does not support a mode to run the game at full speed without also running a graphical interface, and does not support the automated extraction of metrics or other details of the supply chains created.

The second problem is to answer the question \emph{In the game OpenTTD, what are properties of supply chains that are associated with resilience?}. This is a broad question - it will not be able to be answered conclusively in the time available. However, it should be able to be answered in some limited extent, for example by focusing on highly simplified worlds, a single transport mechanism, or simple supply chains of goods, or being robust to very limited events. Exactly what limitations will be applied will be chosen during the project.

\subsection{Research Hypothesis and Objectives}

Ideally, to make OpenTTD useful for reproducible research, any changes will be merged into its source code. And these changes will make it straightforward, so in a very low number of commands or configuration changes, to run experiments over a range of scenarios or AIs. Such steps can be measured - measuring the number of steps or decision points a researcher must take in order to run a set of experiments to produce some research. The ideal is a single step - a researcher running a single command, and results are output.

The second phase of the project has more of a vague objective - to determine properties of supply chains that offer some robustness.

\subsection{Timeliness and Novelty}

While being based on a 1990s computer game, OpenTTD is actively developed and developed. For example, four releases have been made so far in 2023 \cite{openttdReleases}. The study of supply chain resilience is is an increasing area of research, especially following the coronavirus (COVID-19) pandemic that started in 2020 and the 2021 Suez Canal obstruction by the container ship Ever Given. The results of an informal analysis of published research into supply chain resilience into is shown in Figure \ref{fig:supplychainresiliance}.

\begin{figure}[h]
\centering
\begin{gnuplot}[terminal=cairolatex]
set ylabel 'Number of publications'
set xlabel 'Year'
set style data histograms
set style histogram rowstacked
set style fill pattern border -1
set xtics nomirror
set key samplen 5
set key invert
set key left top
plot 'supply-chain-resiliance-research.dat' \
    using 2 t "Not containing COVID-19", \
    '' using 3:xticlabels(1) t "Containing COVID-19"
\end{gnuplot}
\caption{How the number of publications on the topic "supply chain resilaince" has increased between 2014 and 2022.}
\label{fig:supplychainresiliance}
\end{figure}

In terms of novelty, the first phase of the project solves a clear deficiency in OpenTTD - repeatedly running experiments with a parameterised AI for example is a manual process that the phase seeks to solve. In the second phase, OpenTTD does not appear to have been used in the study of resilient transport networks or supply chains.

\subsection{Significance}

The proposal should demonstrate the originality of your intended research. You should therefore explain why your research is important (for example, by explaining how your research builds on and adds to the current state of knowledge in the field or by setting out reasons why it is timely to research your proposed topic) and providing details of any immediate applications, including further research that might be done to build on your findings.

The headless mode of OpenTTD is itself valuable. Even without the AI, this is a contribution - it will others to reproduce any results in this project, and allow further work to be done in a reproducible way.

\subsection{Feasibility}

The fact that OpenTTD is open source allow changes to be made. and the 50 existing AIs suggest its possible to make open AIs using a variety of algorithms. I'm an experienced software engineer with experience of working in unknown codebases, knowing several programming languages, and so it should be feasible for me to make changes to the source code if necessary, and to write an AI in the available time. This makes this project feasible.

However, the project is ambition introduces risks however, which are detailed in section \ref{riskassessment} along with their mitigations.

\subsection{Beneficiaries}

The first phase of the project should benefit researchers wanting to use OpenTTD for experiments. The second phase. In all cases code will be published to GitHub, with clear instructions on how to use it in order to reproduce results. In fact, the ideal of the first phase is to allow researches to reproduce any results in a single command.

What sort of qualities do networks have that are robust?

Describe how the research will benefit other researchers in the field and in related disciplines. What will be done to ensure that they can benefit? 

\section{Background and Related Work}

Describe some existing openttd AIs \cite{openttd}.

Things done in repdocubility, why not suitable

Something to do with robust supply chains?

\section{Programme and Methodology}

This project will be done in 3 phases that I'm calling Preparation, Reproducibility, and Robustness.

\begin{enumerate}
\addtocounter{enumi}{-1}
    \item \textbf{Preparation}

    There will be an initial short preparation phase of this project is make sure I have a dissertation template - will be virtually empty at the beginning, and that I can compile OpenTTD, and run a very basic AI. If I'm unable to do any of these steps, this must be discovered very early on in order to change the later plan.

    As discussed later in this proposal, it is deliberate to construct the dissertation at the beginning rather than a final writing up stage.
    
    \item \textbf{Reproducibility}

    The next phase of the project focuses on making any required changes to OpenTTD in order to support reproducible research. This itself has two parts - a basic headless mode, and the construction of a naive AI developed using the headless mode. The purpose of the second part is to refine the headless mode in terms of usage and output, and to make sure that constructing an AI with certain properties is feasible in the time available.
    
    \item \textbf{Robustness}

    The final phase is to use the results of the previous phases to construct an AI that can be trained to make networks that are resiliant, and compare them against AIs not trained to be resiliant.
    
\end{enumerate}

In terms of methodology, this project will be undertaken in a highly iterative way - tight cycles of development and evaluation. There is suitable since this is an ambitious project with a number of unknowns. This will include maintaining a (mostly) ready to submit dissertation.

\subsection{Risk Assessment}
\label{riskassessment}

Ideally any changes to OpenTTD would be merged into its codebase, and so maintained indefinitely. However, there is no guarantee of this. Although OpenTTD is actively maintained, PRs remain open from 2019, and its maintainers are under no obligation to merge them or any PRs submitted as part of this project. However, there are migitations to this.

\begin{itemize}
    \item Changes will be submitted as early as possible in the project
    \item I'll submit changes that are as small as possible, and so more likely to be merged
    \item Continue to work as though changes will not be merged - the remainder of the project does not need to wait and will still output value without it.
    \item Reflection on what PRs are and are not merged is helpful in it of itself
\end{itemize}

Overall, the biggest risk overall is running out of time without any output - software projects are overdue more often than they are on time. From my own experience as a software engineer, the project where the engineer is not familiar with the domain have the highest risk of this. This is such a project - I am not familiar with the OpenTTD codebase.

To mitigate this risk, I leverage the fact that while the submission deadline is fixed, what is submitted by that deadline is not, allowing me to take a multi-scale agile/iterative approach to the project, adjusting its aim as it progresses. Specifically:

\begin{itemize}
    \item Each phase/iteration is a de-facto feasibility study for the next phase
    \item Each phase/iteration results in value in terms of novelty, timeliness, and significance
    \item Each phase/iteration results will have reasonably complete project at all times
\end{itemize}

To reduce the need to trade-off ambition against leaving time to write up, this includes the deliverable dissertation itself. The dissertation will be started at the beginning of the project rather than the end, and maintained as the project progresses. Figure \ref{fig:gantt} shows how this can look in the style of a more traditional gantt chart, and a summary of these risks are in Table \ref{fig:risks}.

\begin{table}[htbp]
    \begin{center}
        \begin{tabular}{|l|l|l|l|}
        \hline
        \textbf{Phases} & \textbf{Risk} & \textbf{Mitigations} & \textbf{Residual risk} \\
        \hline
        1 & Merges upstream take time & Initiate early in project & Low \\
        1 & No merges upstream & Small frequent changes & Low \\
          &                    & Can maintain fork & \\
        1, 2 & Any phase too time-consuming & Expand previous phases & Low \\
        \hline
        \end{tabular} 
    \end{center}
    \caption[Project milestones]{Risks and mitigations}
    \label{fig:risks}
\end{table}

\subsection{Ethics}

All data will be generated during the project through simulations. Specifically, there will be no data collected from individuals.

What about reflecting if PRs where merged or not?? People do this...

However, there are actions that any AI could be deemed as unethical if applied to the real world. For example, it is possible that an AI destroys buildings while making the network. A set of metrics can be extracted by the first phase of the project.

\section{Evaluation}

The first phase of the project, making it easier for OpenTTD will somewhat subjectively evaluated by myself. However, this is mitigated by an objective component - the number of steps required to be taken in order to reproduce research conducted with OpenTTD.

The second phase of the project. The analysis will be done XXX and intepreseted XX.

\section{Expected Outcomes}

At a minimum, the first phase of the project should make it easier to run experiments in OpenTTD. The second phase would use this output, and refine it, and ideally add to the understanding of supply chains.

\section{Research Plan, Milestones and Deliverables}

\begin{figure}[htbp]
\begin{ganttchart}[
    vgrid,inline,
    x unit=1cm,
    time slot format=isodate-yearmonth,
    time slot unit=month,
    title height=1,
    group peaks height=0,
    group left shift=0,
    group right shift=0,
    group top shift=.7,
    bar height=.6
   ]{2023-05}{2024-08}
    \gantttitlecalendar{year, month=shortname} \\
    \ganttgroup{Prep.}{2023-05}{2023-06} \\
    \ganttbar[name=dissertation, bar label font=\footnotesize]{Dissertation}{2023-05}{2023-06} \\
    \ganttbar[name=dissertation]{Compile}{2023-05}{2023-06} \\
    \ganttgroup{Reproducibility}{2023-06}{2023-11} \\
    \ganttgroup[group height=.1]{Headless mode}{2023-06}{2023-08} \\
    \ganttbar[name=i1]{1}{2023-06}{2023-06} \\
    \ganttbar[name=i2]{2}{2023-07}{2023-07} \ganttlink{i1}{i2} \\
    \ganttbar[name=i3]{3}{2023-08}{2023-08} \ganttlink{i2}{i3}  \\
    \ganttgroup[group height=.1]{Naive AI}{2023-09}{2023-11} \\
    \ganttbar[name=i4]{4}{2023-09}{2023-09} \ganttlink[link mid=.25]{i3}{i4} \\
    \ganttbar[name=i5]{5}{2023-10}{2023-10} \ganttlink{i4}{i5}   \\
    \ganttbar[name=i6]{6}{2023-11}{2023-11} \ganttlink{i5}{i6}   \\  
    \ganttgroup{Robust AI}{2023-12}{2024-08} \\
    \ganttbar[name=i7]{7}{2023-12}{2023-12} \ganttlink[link mid=.25]{i6}{i7}  \\
    \ganttbar[name=i8]{8}{2024-01}{2024-01} \ganttlink{i7}{i8}  \\
    \ganttbar[name=i9]{9}{2024-02}{2024-02} \ganttlink{i8}{i9}  \\
    \ganttbar[name=i10]{10}{2024-03}{2024-03} \ganttlink{i9}{i10}  \\
    \ganttbar[name=i11]{11}{2024-04}{2024-04} \ganttlink{i10}{i11}  \\
    \ganttbar[name=i12]{12}{2024-05}{2024-05} \ganttlink{i11}{i12}  \\
    \ganttbar[name=i13]{13}{2024-06}{2024-06} \ganttlink{i12}{i13}  \\
    \ganttbar[name=i14]{14}{2024-07}{2024-07} \ganttlink{i13}{i14}  \\
    \ganttbar[name=i15]{15}{2024-08}{2024-08} \ganttlink{i14}{i15}  
\end{ganttchart}
\caption[Project Gantt chart]{Gantt Chart of the activities defined for this project. This project will be undertaken on a part time basis and in a highly iterative way. Each iteration will contain reading, development, evaluation, and writing up. Each will result in complete project, albeit with limited scope.}
\label{fig:gantt}
\end{figure}

\begin{table}[htbp]
    \begin{center}
        \begin{tabular}{|c|S[table-format=2.0]|l|}
        \hline
        \textbf{Milestone} & \textbf{Week} & \textbf{Description} \\
        \hline
        $M_1$ & 2 & Feasibility study completed \\
        $M_2$ & 5 & First prototype implementation completed \\
        $M_3$ & 7 & Evaluation completed \\
        $M_4$ & 10 & Submission of dissertation \\
        \hline
        \end{tabular} 
    \end{center}
    \caption[Project milestones]{Milestones defined in this project.}
    \label{fig:milestones}
\end{table}

\begin{table}[htbp]
    \begin{center}
        \begin{tabular}{|c|S[table-format=2.0]|l|}
        \hline
        \textbf{Deliverable} & \textbf{Week} & \textbf{Description} \\
        \hline
        $D_1$ & 6 & Headless mode for to OpenTTD \ldots\\
        $D_2$ & 8 & Evaluation report on \ldots\\
        $D_3$ & 10 & Dissertation \\
        \hline
        \end{tabular} 
    \end{center}
    \caption[Project deliverables]{List of deliverables defined in this project.}
    \label{fig:deliverables}
\end{table}


%                Now build the reference list
\bibliographystyle{unsrt}   % The reference style
%                This is plain and unsorted, so in the order
%                they appear in the document.

{\small
\bibliography{main}       % bib file(s).
}
\end{document}

